\svnidlong
{$HeadURL: $}
{$LastChangedDate: $}
{$LastChangedRevision: $}
{$LastChangedBy: $}

\odachapter{Writing your own wrapper}

\begin{tabular}{p{4cm}l}
\textbf{Contributed by:} & vacancy\\
\textbf{Last update:}    & \svnfilemonth-\svnfileyear\\
\end{tabular}

\section{Te schrijven hoofdstuk over model-wrappers.}

\begin{itemize}
\item Tips en tricks voor het schrijven van een wrapper klasse. Leg uit dat ook
  hier dingen beschikbaar gemaakt kunnen worden die mogelijk niet in de uitvoer
  zitten. Je kan immers altijd extra (output) exhangeItems maken voor
  bijvoorbeeld -speciale punten -omrekenen van grootheden -beschikbaar stellen
  van gemiddelden -etc
\item Wat uitleg over de wrapper klasse. Wat moet die ongeveer doen. Laten zien
  dat je een implementie meerdere keren kan gebruiken, normaal een klasse per
  bestand.
\item IOOBject implementeren voor je model (DLL koppeling)
\end{itemize}
